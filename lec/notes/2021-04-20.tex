\documentclass[12pt, leqno]{article} %% use to set typesize
\input{common}

\begin{document}
\hdr{2021-04-20}

\section{Graphs and linear algebra}

Formally, an unweighted graph is $\calG = (\calV, \calE)$, where
$\calE \subset \calV \times \calV$.  Informally, $\calV$ consists of
things we want to model and $\calE$ represents the relations between
them.  It is a very flexible representation: we use graphs to
represent friendships between people, wires between routers, citations
between papers, links between objects in a data structure, and many
other things.  When the bare topology of the relationships does not
provide enough modeling power, we might also consider including
functions on $\calV$ or $\calE$ corresponding to different
attributes.  The most common case is a scalar {\em weight} function
assigned to each edge that corresponds to the importance of the
relation: in a social network, for example, maybe a close and active
friendship has more weight than a casual acquaintance.
We use relationship information encoded in graphs to reason about
logical groupings (whether we call them communities or clusters),
about power relations and influence, and about dynamic processes like
the spread of rumors or disease.

Matrix methods for network analysis rely on a sort of pun: we
encode the network as a matrix, translate the question into linear
algebraic terms%
\footnote{``Mathematicians are like Frenchmen: whatever you say to
  them they translate into their own language and forthwith it is
  something entirely different.'' -- Goethe},
and commence to compute.  There are many possible matrices
associated with a graph, and we use them to reason about different
things.  We consider three interpretations of these network matrices:
\begin{itemize}
\item
  A matrix may represent a {\em linear map} between {\em different spaces},
  typically mapping vertex properties to edge properties, or
  vice-versa.  Examples include the discrete gradient operator and the
  edge sum operator.
\item
  A matrix may represent an {\em operator} mapping the space of
  functions over the vertices (or over edges) to itself.  Examples
  include transition matrices for random walks defined on the graph.
\item 
  A matrix may represent a {\em quadratic form} mapping functions
  on the vertices (or edges) into scalars.  Often the quadratic form
  has an easy to interpret meaning for special inputs; for example,
  the quadratic form for the {\em combinatorial Laplacian}
  counts cut edges in graph partitioning.
\end{itemize}

\section{Adjacency and degrees}


If we identify vertices in the graph with indices $1, \ldots, n$, then
the (unweighted) {\em adjacency matrix} $A \in \bbR^{n \times n}$ has entries
\[
  a_{ij} =
  \begin{cases}
    1, & (i,j) \in \calE \\
    0, & \mbox{otherwise}.
  \end{cases}
\]
For graphs in which edges have positive weights, we sometimes use the
{\em weighted} adjacency matrix, with $a_{ij}$ giving the edge weight
for $(i,j) \in \calE$.

The {\em degree} $d_i$ of a node is the sum of the weights on the
incident edges.  When the graph is directed, the in-degree and
out-degree may differ; for the moment, we will stick with the directed
case.  We let $d \in \bbR^n$ denote the vector of weighted node
degrees, and let $D$ denote the matrix in which the weighted node
degrees appear on the diagonal.

The adjacency matrix and the degree matrices are building blocks for
several other matrices, but they are also useful on their own.
First, as a linear operator, the adjacency matrix accumulates values
from neighbors; that is,
\[
  (Ax)_i = \sum_{j \in N_i} x_j
\]
where $N_i = \{ j : (i,j) \in \calE \}$ is the neighborhood of $i$.
If $x_j$ is the number of paths of length $k$ leading from starting
point to node $j$, then $(Ax)_i$ is the number of paths of length $k$
to all neighbors of node $i$ --- that is, the total number of paths of
length $k+1$ to node $j$.  Therefore,
\[
  [A^k]_{ij} = \mbox{number of paths of length $k$ from $i$ to $j$}.
\]
We use this formula in many ways; for example, it lets us
write number of triangles in an undirected graph (closed
cycles of length three) as
\[
  \mbox{number of triangles}
    = \frac{1}{3} \sum_{i} [A^3]_{ii}
    = \frac{1}{3} \tr(A^3)
\]
where we divide by three because each triangle is counted once for
each of its vertices.  We also know, for example, how to approximate
the number of long paths between $i$ and $j$ in terms of the dominant
eigenvector (and associated eigenvalue) of $A$.

The matrix $A$ also defines a quadratic form that is useful for
counting edges.  Let $x \in \{0,1\}^n$ be the indicator for a subset
of vertices $S \subset \calV$.  Then
\[
  x^T A x
  = \sum_{i,j} a_{ij} x_i x_j
  = \sum_{(i,j) \in S \times S} a_{ij},
\]
i.e.~$x^T A x$ is the total (directed) edge weight between nodes in
$S$, or twice the total undirected edge weight.
If $x$ is an indicator then $x^T x = |S|$, and so
\[
  \frac{x^T A x}{x^T x} = \mbox{mean degree within } S.
\]
If $S$ is a clique, the mean degree within $S$ is $|S|-1$; therefore
\[
  |S|-1 = \frac{x^T A x}{x^T x} = \rho_A(x) \leq \lambda_{\max}(A),
\]
since $\lambda_{\max}(A)$ is the largest possible value for the
Rayleigh quotient $\rho_A(x)$.  Hence, the maximum clique size
$k(\calG)$ has the bound
\[
  k(\calG) \leq 1 + \lambda_{\max}(A).
\]
This is an example of a result in
{\em spectral graph theory}, i.e.~the study of graphs in terms of
eigenvalues and eigenvectors of associated matrices.  In fact,
another continuous optimization problem due to Motzkin and Straus
gives the clique number exactly:
\[
  1-1/k(\calG) = \max_{x \in \Delta_n} x^T A x, \mbox{ where }
  \Delta_n \equiv \{ x \in \bbR^n : x \geq 0, e^T x = 1 \}.
\]
The optimization is now carried out over the simplex $\Delta_n$
rather than over the Euclidean unit ball used in the spectral bound.

If $x$ is an indicator for a set $S$, we can also use the
quadratic form
\[
  x^T D x = \sum_{i \in S} d_i = \mbox{edges incident on } S.
\]
Therefore,
\[
  x^T D x - x^T A x = x^T (D-A) x = \mbox{edges between } S \mbox{ and } S^C.
\]
We will see more of the {\em combinatorial Laplacian} matrix
$L = D-A$ shortly.

\section{Random walks and normalized adjacency}

Now consider a random walk on a $\calG$, i.e.~a Markov
process where the walker location $X^{t+1}$ at time $t+1$ is chosen
randomly from among the neighbors of the previous location $X^t$ with
probability determined by the edge weights.  Using the properties of
conditional probability,
\[
  P\{X^{t+1} = i\} = \sum_j P\{X^{t+1} = i | X^t = j\} P\{X^t = j\},
\]
and the rule for randomly choosing a neighbor gives
\[
  P\{X^{t+1} = i | X^t = j\} = \frac{a_{ij}}{d_j}.
\]
Letting $\pi^{t+1} \in \bbR^n$ be the column vector whose entries represent the
probability that $X^{t+1} = i$, and similarly with $\pi^t$,
we write the equation for conditional probability concisely as
\[
  \pi^{t+1} = (AD^{-1}) \pi^t.
\]
The matrix $T = AD^{-1}$ is the {\em transition matrix} for the random walk
Markov chain\footnote{%
  We use the convention that probability densities are column
  vectors, and that $a_{ij}$ represents a transition from $j$ to $i$.
  If $\calG$ is directed, we also denote by $D$ the out-degree of the nodes.
  This is consistent with the conventions in numerical linear algebra;
  in other areas, probability densities are typically rows.
}.
Powers of $T$ have an interpretation similar to that of powers of $A$,
but rather than counting length $k$ paths, $T^k$ computes
probabilities:
\[
  [T^k]_{ij} = P\{ X^{k} = i | X^{0} = j \}.
\]
Assuming the graph is connected and and aperiodic\footnote{%
  The graph is aperiodic if there is some $k$ such that there is a
  length $k$ path between any nodes $i$ and $j$.
},
the matrix $T$ has a unique eigenvalue at 1, and all other eigenvalues
are inside the unit circle.  In this case,
\[
  \lim_{k \rightarrow \infty} T^k = T^{\infty} = (\pi^*) e^T
\]
where $\pi^*$ is a probability vector representing the
{\em stationary distribution} for the Markov chain.  In the undirected
case, the stationary distribution is rather simple: $\pi^*_i =
d_i/(2m)$.  Things are more interesting for directed graphs.

While the eigenvalue at $1$ and the associated stationary distribution
are particularly interesting, the other eigenvalues and vectors are
also interesting.  In particular, suppose $T = V \Lambda V^{-1}$,
and consider
\[
  \|T^k-T^\infty\| = \|V \bar{\Lambda}^k V^{-1} \| \leq \kappa(V) |\lambda_2|^k,
\]
where $\kappa(V) = \|V\| \|V^{-1}\|$ is the condition number of the
eigenvector matrix, $\bar{\Lambda}$ is the diagonal matrix of
eigenvalues with the eigenvalue at one replaced by zero, and $|\lambda_2|$
is the maximum modulus of all eigenvalues other than the eigenvalue at one.
Therefore, the asymptotic rate of convergence of the Markov chain,
also known as the {\em mixing rate} is determined by the
second-largest eigenvalue modulus of $T$.

To understand the mixing rate in more detail (in the undirected case),
it is helpful to consider the closely related
{\em normalized  adjacency matrix}
\[
  \bar{A} = D^{-1/2} A D^{-1/2}.
\]
Note that $\bar{A} = D^{-1/2} T D^{1/2}$, so
\[
  (v, \lambda) \mbox{ an eigenpair of } \bar{A} \iff
  (D^{1/2} v, \lambda) \mbox{ an eigenpair of } T.
\]
The eigenvalues of $\bar{A}$ are critical points of
\[
  \rho_{\bar{A}}(x) = \frac{x^T D^{-1/2} A D^{-1/2} x}{x^T x};
\]
substituting $x = D^{1/2} y$, we have
\[
  \rho_{\bar{A}}(D^{1/2} y) = \rho_{(A,D)}(y) = \frac{y^T A y}{y^T D y}.
\]
If $y$ is an indicator for a set $S$, this last expression represents
the fraction of edges incident on $S$ that are to other nodes in $S$.
If $z = 2y-e$ is $+1$ on $S$ and $-1$ on $S^c$, then $z^T D z = 2m$
and $z^T A z$ is $2m$ minus twice the total weight $|C(S)|$ of
edges from $S$ to $S^C$; hence,
\[
  \frac{z^T A z}{z^T D z} = 1-\frac{|C(S)|}{m}
\]
If we restrict to the case where the same number of edges
are incident on $S$ and $S^C$, then $z$ is $D$-orthogonal
to the all one vector, and so $\rho_{(A,D)}(z)$ is a lower
bound on the eigenvalue closest to one.  Thus, spectral analysis
lets us bound the mixing rate in terms of the normalized cut
size $|C(S)|/m$.

\section{Discrete gradients and the Laplacian}

For an unweighted graph, the {\em discrete gradient}
$G \in \bbR^{m  \times n}$ is a matrix in which each row represents an edge
$(i,j) \in \calE$ by $(e_i-e_j)^T$.  If $x$ is an indicator for a set $S$,
then $Gx$ is nonzero ($\pm 1$) only on edges between $S$ and $S^c$;
hence,
\[
  \|Gx\|^2 = \mbox{edges between } S \mbox{ and } S^C.
\]
We can rewrite this as $x^T G^T G x = x^T L x$ where
\[
  L = \sum_{(i,j) \in \calE} (e_i-e_j) (e_i-e_j)^T.
\]
Each term in this sum contributes one to the $l_{ii}$ and $l_{jj}$
entries through the $e_i e_i^T$ and $e_j e_j^T$ products; the cross
terms fill in $l_{ij} = l_{ji} = -1$.  Putting everything together,
we have
\[
  L = D-A.
\]
The matrix $L$ is known as the {\em combinatorial Laplacian},
or sometimes simply as the Laplacian.  The same construction holds
in the weighted case, where it corresponds to
\[
  L = \sum_{(i,j) \in \calE} a_{ij} (e_i-e_j) (e_i-e_j)^T
\]
where $a_{ij}$ is the weight of the $(i,j)$ edge.  In either case,
the smallest eigenvalue of $L$ is zero, corresponding to an
eigenvector of all ones.  The multiplicity of the zero eigenvalue
is equal to the number of connected components in the graph;
assuming there is only one connected component, the second largest
eigenvalue $\lambda_2$ is a lower bound on $x^T L x$ for any $x^T e = 0$;
if we choose $x$ to be a $\pm 1$ vector indicating the split between
equal size sets $S$ and $S^c$, then $x^T L x$ also gives four times
the number of edges cut by the partitioning.  Hence, $\lambda_2/4$
is a lower bound on the minimal bisector size.

The combinatorial Laplacian also can be interpreted as a linear
operator, and in this guise it plays a role as the generator for a
{\em continuous-time random walk} involving a random walk in which
the time elapsed between each consecutive pair of steps is given by
an independent exponential random variable with mean one.  In this
case, we have
\[
  \exp(-sL)_{ij} = P\{X(s) = i | X(0) = j\}.
\]
The matrix $\exp(-sL)$ is known as the {\em heat kernel} on the graph
because it can also describe continuous-time diffusion of heat on
a graph.

The {\em normalized} Laplacian is $\bar{L} = D^{-1/2} L D^{-1/2} = I-\bar{A}$;
the eigenvalues of the normalized Laplacian can also be expressed
as critical points of the generalized Rayleigh quotient
\[
  \rho_{(L,D)}(x) = \frac{x^T L x}{x^T D x}.
\]
Thus twice the Rayleigh quotient gives the fraction of all edges
that go between $S$ and $S^C$ if $x$ is a vector which is $+1$ on the
set $S$ and $-1$ on $S^c$.

\section{Discrete sums and the signless Laplacian}

The {\em discrete sum} operator is $G^+ \in \bbR^{m \times n}$ where
each row corresponds to an edge $(i,j) \in \calE$ and has the form
$(e_i + e_j)^T$.  The {\em signless Laplacian} is
\[
  L^+ = D + A = (G^+)^T (G^+) = \sum_{(i,j) \in \calE} (e_i + e_j) (e_i + e_j)^T.
\]
The signless Laplacian is positive semi-definite; but unlike the
combinatorial Laplacian, it may or may not have a null vector.  If
there is a null vector $x$ for the signless Laplacian, then we must
have $x_i = -x_j$ whenever $(i,j) \in \calE$; this implies that $x$
indicates {\em bipartite structure} where the set $S$ with positive
elements can have edges to a set $S'$ with negative elements,
but neither $S$ nor $S'$ may have any other edges.  There has been
some work on the spectral theory for the signless Laplacian, but it
is generally much less used than the combinatorial Laplacian.

\section{Modularity matrix}

Suppose we want to find a tight cluster in our graph.  One approach is
to look for sets of nodes that make $x^T A x$ large; but large
relative to what?  For an undirected graph, we use the quadratic form
$x^T A x$ to count internal edges in a set.  But a set may have many
internal edges purely as an accident of having many high-degree nodes:
high-degree nodes are highly likely to connect to each other simply
because they are highly likely to connect to {\em anyone}!  Hence,
we would like a reference model so that we can see whether the edge
density in a subgraph is ``unusually'' high relative to that
reference.

The simplest reference model that takes into account the effect of
degree distribution is sometimes called the {\em configuration model}.
In the configuration model, we prescribe the vector $d$ of expected
node degrees.  We then add $m$ edges, each of which is $(i,j)$ with
probability $d_i d_j/(2m)^2$; self-loops and repeating edges are
allowed.  With this construction, the expected adjacency is
\[
  \bar{A} = \frac{dd^T}{2m},
\]
which has the correct expected degree distribution.  
The {\em modularity} matrix is defined to be
\[
  B = A-\bar{A},
\]
and if $x$ is an indicator for $S$, the quadratic form $x^T B x$
indicates the number of ``excess edges'' within $S$ compared to what
we would predict in the configuration model.

\section{And many more}

The list of graph matrices that we have discussed is by no means
exhaustive.  We will see a few more examples this week, including the
heat kernel matrix and the PageRank matrix.  But there are many more
besides; for example, recent work by Leskovec and Benson used a
{\em motif adjacency matrix} $M = (A \odot A) A$ for which $m_{ij}$
represents the number of triangles in the graph involving the edge $(i,j)$.
And one can define ever more exotic matrix representations.
However, the adjacency, Laplacian, and their close neighbors suffice
for many applications.

\end{document}
