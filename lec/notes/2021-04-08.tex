\documentclass[12pt, leqno]{article} %% use to set typesize
\input{common}

\begin{document}
\hdr{2021-04-08}

\section{Regularization in kernel methods}

Last time, we discussed kernel methods for {\em interpolation}:
given $f_X$, we seek an approximation $\hat{f}_X$ such that
$\hat{f}_X = f_X$.  However, the kernel matrix $K_{XX}$ is sometimes
very nearly ill-conditioned, particularly when the underlying kernel
is smooth (e.g.~a squared exponential kernel).  This near-singularity
means that when $f$ is not smooth enough to lie in the native space
$\mathcal{H}$ for the kernel, or when our measurements are
contaminated with some error, kernel interpolation schemes may not
give accurate results.  And even when $f$ is in the appropriate native
space and we are able to evaluate $f$ exactly, this lack of stability
may cause problems because of the influence of rounding errors.

We deal with the problem of instability in kernel methods the same way
we deal with instability in other fitting problems: we
{\em regularize}.  There are many ways that we might choose to
regularize, but we will focus on the common Tikhonov regularization
approach.  As always in kernel methods, there are multiple stories for
the same method; we will tell two of them.

\subsection{Feature space and kernel ridge regression}

Recall the feature space version of kernel interpolation: write
\[
\hat{f}(x) = \psi(x)^T c
\]
where $c$ is determined by the problem
\[
  \mbox{minimize } \|c\|^2 \mbox{ s.t.~} \Psi^T c = f_X
\]
with $\Psi$ the matrix whose columns are feature vectors at the data
points in $X$.
{\em Kernel ridge regression} instead solves the unconstrained
minimization problem
\[
  \mbox{minimize } \lambda \|c_{\lambda}\|^2 + \|\Psi^T c_{\lambda} - f_X\|^2
\]
That is, rather than enforcing the interpolation constraints, we
minimize a combination of a regularity term (the norm of
$c_{\lambda}$) and a data fidelity term.  As the weight $\lambda$ goes
to zero, we recover kernel interpolation.  For nonzero $\lambda$,
we solve the critical point equations
\[
  \lambda c_{\lambda} + \Psi (\Psi^T c_{\lambda} - f_x) = 0,
\]
which we may rewrite using $r = \Psi^T c_\lambda -f_X$ as
\[
  \begin{bmatrix}
    \lambda I & \Psi \\
    \Psi^T & -I
  \end{bmatrix}
  \begin{bmatrix} c_\lambda \\ r \end{bmatrix} =
  \begin{bmatrix} 0 \\ f_X \end{bmatrix}.
\]
Eliminating $c_{\lambda}$ gives the equation
\[
  -(I+\lambda^{-1} \Psi^T \Psi) r = f_X,
\]
and back-subsittution yields
\[
  \lambda c_{\lambda} = \Psi (I+\lambda^{-1} \Psi^T \Psi)^{-1} f_X.
\]
Dividing both sides by $\lambda$, we have
\[
  \hat{f}_{\lambda}(x) = \psi(x)^T c_{\lambda} =
    \psi(x)^T \Psi (\Psi^T \Psi + \lambda I)^{-1} f_X,
\]
and applying the kernel trick gives
\[
  \hat{f}_{\lambda}(x) = k_{xX} (K_{XX} + \lambda I)^{-1} f_X.
\]

As with kernel interpolation, the story of kernel ridge regression can
be told without reference to a particular basis or feature map.
Observe as before that $\Psi^T c_{\lambda} = \hat{f}_{\lambda,X}$ and that
$\|c_{\lambda}\| = \|\hat{f}\|_{\mathcal{H}}^2$ for an appropriate
reproducing kernel Hilbert space.  The kernel ridge regression problem
is therefore
\[
  \mbox{minimize } \lambda \|\hat{f}\|_{\mathcal{H}}^2 + \|\hat{f}_X -f_X\|^2
  \mbox{ over } \hat{f} \in \mathcal{H}.
\]

\subsection{GPs with noise}

The interpretation of Tikhonov regularization for Gaussian processes
is straightforward.  Suppose that $f$ is drawn from a GP with mean
zero and covariance kernel $K$, and we wish to compute a marginal
distribution conditioned on knowing $y = f_X + u$ where
$u$ is a vector of independent Gaussian random variables with zero
mean and variance $\sigma^2$.  Then the posterior distribution
for $f$ conditioned on the data is a GP with mean and covariance
\begin{align*}
  \hat{\mu}(x) &= k_{xX} \tilde{K}^{-1} f_X, &
  \hat{k}(x,x') &= k(x,x') - k_{xX} \tilde{K}^{-1} k_{Xx'}
\end{align*}
where $\tilde{K} = K + \sigma^2 I$.  The derivation comes from looking
at the multivariate Gaussian distribution of the data ($y$) together
with function values at locations of interest.

\section{Choosing hyperparameters}

Whether we call it kernel ridge regression or GP regression with
noise, Tikhonov regularization involves a free parameter --- which may
come in addition to other hyper-parameters in the kernel, like length
scale or scale factor.  How do we choose all these hyper-parameters?
There are several methods, though all sometimes fail, and none of
them is uniformly the best.  We will discuss four approaches:
the discrepancy principle, the L-curve, cross-validation, and maximum
likelihood estimation.

\subsection{The discrepancy principle and the L-curve}

The {\em discrepancy principle} (due to Morozov) says that we should
choose the regularization parameter based on the variance of the
noise.  This may be useful when the primary source of error comes from
measurement noise from instruments (for example), but often we do not
have this information; what shall we do then?  Also, the discrepancy
principle does not tell us what to do with other hyper-parameters,
such as kernel length scales.

The {\em L-curve} is a graphical plot on a log-log axis of the norm of
the data fidelity term versus the norm of the regularization term.
Often such plots have a ``corner'' associated with the favored choice
$\lambda_*$ for the regularization parameter.  Increasing $\lambda$
beyond $\lambda_*$ increases the residual error quickly, while
decreasing $\lambda$ from $\lambda_*$ has only a modest impact on the
residual error, but causes a rapid increase in the size of the
regularization term.  As with the discrepancy principle, the L-curve
is primarily used for determining a single regularization parameter,
and not for fitting other hyper-parameters such as the kernel
length-scale.

\subsection{Cross-validation}

The idea of {\em cross-validation} is to fit the method to split the
training data into two sets: a subset that we actually use to fit the
model, and a held-out set to test the generalization error.  We
usually use more than one splitting of the data to do this.  For
example, the {\em leave-one-out cross-validation} (LOOCV) statistic
for a regression method for a function $f$ on data points $X$ is
\[
  \mbox{LOOCV} = \frac{1}{n} \sum_{i=1}^n (f^{(-i)}(x_i)-f(x_i))^2
\]
where $f^{(-i)}$ refers to the model fit to all of the points in $X$
except for $x_i$.  One can do more complicated things, but the LOOCV
statistic has a lovely structure that lets us do fast computations,
and this is worth exploring.

\subsubsection{Fast LOOCV for least squares}

Before we explore the case of kernel methods, let us first consider
LOOCV for the ordinary linear least squares problem:
\[
  \mbox{minimize } \|Ax-b\|^2
\]
To compute the LOOCV statistic in the most obvious way, we would
delete each row $a_i^T$ of $A$ in turn, fit the model coefficients
$x^{(-i)}$, and then evaluate $r^{(-i)} = b_i - a_i^T x^{(-i)}$.
This involves $m$
least squares problems, for a total cost of $O(m^2 n^2)$ (as opposed
to the usual $O(mn^2)$ cost for an ordinary least squares problem).
Let us find a better way!

The key is to write the equations for $x^{(-i)}$ as a small change to
the equations for $A^T A x^* = A^T b$:
\[
  (A^T A - a_i a_i^T) x^{(-i)} = A^T b - a_i b_i.
\]
This subtracts the influence of row $i$ from both sides of the normal
equations.  By introducing the auxiliary variable $\gamma = -a_i^T x^{(-i)}$,
we have
\[
  \begin{bmatrix}
    A^TA & a_i \\
    a_i^T & 1
  \end{bmatrix}
  \begin{bmatrix} x^{(-i)} \\ \gamma \end{bmatrix} =
  \begin{bmatrix} A^T b - a_i b_i \\ 0 \end{bmatrix}.
\]
Eliminating $x^{(-i)}$ gives
\[
  (1-\ell_i^2) \gamma = \ell_i^2 b_i - a_i^T x^*
\]
where $\ell_i^2 = a_i^T (A^T A)^{-1} a_i$ is called the
{\em leverage score} for row $i$.  Now, observe that
if $r = b-Ax^*$ is the residual for the full problem, then
\[
(1-\ell_i^2) r^{(-i)}
  = (1-\ell_i^2) (b_i + \gamma)
  = (1-\ell_i^2) b_i + \ell_i^2 b_i - a_i^T x_*
  = r_i,
\]
or, equivalently
\[
  r^{(-i)} = \frac{r_i}{1-\ell_i^2}.
\]
We finish the job by observing that $\ell_i^2$ is the $i$th diagonal
element of the orthogonal projector $\Pi = A(A^TA)A^{-1}$, which we
can also write in terms of the economy QR decomposition of $A$ as
$\Pi = QQ^T$.  Hence, $\ell_i^2$ is the squared row sum of $Q$ in
the QR factorization.

\subsubsection{Fast LOOCV for kernels}

The trick for computing the LOOCV statistic for kernels is similar to
the trick for least squares, at least in broad outlines.  Let $c$ be
the coefficient vector fit to all the nodes, and let $c^{(-i)}$ be the
coefficient vector for the expansion fit to all the nodes except node
$i$; that is, we want $c^{(-i)}_i = 0$ and we allow $r^{(-i)} = f(x_i)
- k_{xX} c^{(-i)}$ to be nonzero.  Then
\[
  \begin{bmatrix} \tilde{K} & e_i \\ e_i^T & 0 \end{bmatrix}
  \begin{bmatrix} c^{(-i)} \\ r^{(-i)} \end{bmatrix} =
  \begin{bmatrix} f_X \\ 0 \end{bmatrix},
\]
and Gaussian elimination of $c^{(-i)}$ yields
\[
  [\tilde{K}^{-1}]_{ii} r^{(-i)} = e_i^T \tilde{K}^{-1} f_X = c_i,
\]
and therefore
\[
  r^{(-i)} = \frac{c_i}{[\tilde{K}^{-1}]_{ii}}.
\]
The observent reader may notice that this yields essentially the same
argument we saw in the error analysis of kernel methods, and that
$[\tilde{K}^{-1}]_{ii}^{-1}$ is the squared power function for
evaluating the error at $x_i$ given data at all the other points.

What about the derivatives of $r^{(-i)}$ with respect to any
hyper-parameters?  After all, these are important if we are going to
optimize.  We know that
\[
  \delta[\tilde{K}^{-1}] = -\tilde{K}^{-1} [\delta \tilde{K}] \tilde{K}^{-1}
\]
and differentiating $c = K^{-1} f_X$ (and using the fact that the
function values are independent of the hyper-parameters) gives
\[
  \delta c = -\tilde{K}^{-1} [\delta \tilde{K}] c.
\]
Let $w$ denote $\tilde{K}^{-1} e_i$; then
\begin{align*}
  \delta c_i &= -w^T [\delta \tilde{K}] c, &
  \delta\left( [K^{-1}]_{ii} \right) &= -w^T [\delta \tilde{K}] w,
\end{align*}
and the quotient rule, together with a little algebra, gives
\[
\delta r^{(-i)} = \frac{[\delta \tilde{K} w]^T (wc_i-cw_i)}
                      {[\tilde{K}^{-1}]_{ii}^2}.
\]

\subsection{Maximum likelihood estimation}

Finally, we consider the {\em maximum likelihood estimation} scheme.
If we have data $y$ drawn from a distribution
$p(y ; \theta)$ where $\theta$ are unknown parameters, the idea of
maximum likelihood is to maximize $p(y; \theta)$ with respect to $\theta$.
Often the probability density is awkwardly scaled for computation,
and so we typically instead use the {\em log} likelihood
\[
  \mathcal{L}(\theta) =  \log p(y; \theta)
\]
In the case of Gaussian processes, we have
\[
p(y) = \frac{1}{\sqrt{\det(2\pi K)}}
       \exp\left( -\frac{1}{2} (y-\mu)^T K^{-1} (y-\mu) \right)
\]
and
\[
\log p(y) = -\frac{1}{2} (y-\mu)^T K^{-1} (y-\mu)
            -\frac{1}{2} \log \det(K) - \frac{n}{2} \log(2\pi)
\]
The first term is the {\em model fidelity} term; it is larger the closer
$y$ is to $\mu$ in the norm induced by $K^{-1}$, with a maximum value
of zero when $y=\mu$.  The second term is the {\em model complexity}
term; it is larger when $K$ has lower volume, i.e.~the likely model
predictions are in a smaller region.  The last term is independent
of any kernel hyper-praameters, and so is irrelevant for optimization.

With an eye to optimization, we again want to compute derivatives.
The derivative of the model fidelity term with respect to kernel
hyperparameters is straightforward:
\[
  \delta\left[ -\frac{1}{2} (y-\mu)^T K^{-1} (y-\mu) \right] =
  \frac{1}{2} c^T [\delta K] c
\]
where $c = K^{-1}(y-\mu)$.  The more interesting piece is the
derivative of the log determinant.  To get this, we observe that
\[
  \det(I+E) = \prod_{i=1}^n (1+\lambda_i(E)),
\]
and if $E$ is small, linearization about $E = 0$ gives
\[
  \det(I+E) = 1 + \sum_{i=1}^n \lambda_i(E) + O(\|E\|^2).
\]
Therefore, the derivative of the determinant at the identity
in a direction $E$ is just $\tr(E) = \sum_i \lambda_i(E) = \sum_i E_{ii}$.
We deal with the more general case by observing that
$\det(K+E) = \det(K) \det(I+K^{-1} E)$; hence,
\[
  \delta [\det(K)] = \det(K) \tr(K^{-1} \delta K).
\]
Finally, we have
\[
  \delta[\log \det(K)] = \frac{\delta[\det(K)]}{\det(K)}
    = \tr(K^{-1} \delta K).
\]

\end{document}
