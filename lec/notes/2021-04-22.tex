\documentclass[12pt, leqno]{article} %% use to set typesize
\input{common}

\begin{document}
\hdr{2021-04-22}

\section{Semi-supervised learning}

Suppose we have a collection of objects that we want to classify
one of two ways.  Given some labeled examples, how should we label
the remaining objects?  This is a standard
{\em semi-supervised learning} task.  Of course, labels alone do
not help us unless we have some idea how the objects are related
to each other.  In this lecture, we will assume that this information
comes in the form of a weighted graph, where the objects to be
classified are vertices and the edge weights represent the
degree of similarity or connectedness.  Our problem, then,
is to label the remaining objects so that --- as much as possible ---
similar objects will share the same label.

Before writing methods, we will first introduce some notation.
We will start with the two-class case, and turn to the multi-class
problem later. Let $x$ be the vector of class labels; ideally, we would like
$x \in \{0,1\}^n$.  We order the vertices so that the labeled
examples appear last, and partition $x$ into unlabeled and labeled
subvectors:
\[
  x = \begin{bmatrix} u \\ y \end{bmatrix},
\]
where $u \in \{0,1\}^{n_u}$ is unknown and $y \in \{0,1\}^{n_y}$ is
known.  We let the weighted adjacency matrix $A$ encode the
similarity, and let $L = D-A$ be the weighted Laplacian.  To measure
the quality of a class assignment, we look at the quadratic
\[
  x^T L x
  = \sum_{(i,j) \in \mathcal{E}} a_{ij} (x_i-x_j)^2
\]
which, for 0-1 vectors, gives the total weight of all between-class
edges.  We partition $L$ and $A$ conformally with the partitioning
of $x$:
\[
  L = \begin{bmatrix} L_{uu} & L_{uy} \\ L_{yu} & L_{yy} \end{bmatrix}.
\]
Then we have
\[
  x^T L x = u^T L_{uu} u + 2 u^T L_{uy} y + y^T L_{yy} y.
\]
Alas, optimizing this function with respect to the class assignments $u$
is a challenging discrete optimization problem.

\section{Soft labels}

The optimization is easier if we relax the problem, replacing binary
class labels with real-valued soft labels.  Then we have a continuous
quadratic optimization for which the critical point equation is
\[
  [Lx]_u = L_{uu} u + L_{uy} y = 0.
\]
The matrix $L$ is positive semi-definite, with null vectors that are
constant on each connected component; but we will assume that we have
at least one labeled example in each connected component, so that
$L_{uu}$ is nonsingular.

It is worth looking at the scalar equations in order to understand
this system in more detail.  Let us write row $i$ of the critical
point equations as
\[
  d_i x_i - \sum_{j=1}^n a_{ij} x_j = 0,
\]
and rearrange to find
\[
  x_i = \sum_{j=1}^n \frac{a_{ij}}{d_i} x_j.
\]
The weights $a_{ij}/d_i$ are non-negative and sum to one, so this
tells us that for each $i$ where the label is unknown, we are choosing
$x_i$ to be the {\em weighted average} of the neighbor labels.
This tells us that (for example) all of the computed soft labels will
be in the interval $[0,1]$.

The averaging interpretation of the equlibrium of the equation
suggests an algorithm for computing the soft labels by interpreting
the averaging operation as an update equation:
\[
  x_i^{\mathrm{new}} = \sum_{j=1}^n \frac{a_{ij}}{d_i} x_j.
\]
Several classical {\em point relaxation} methods of
numerical linear algebra follow this approach, differing in the
order in which the updates are computed and applied.  Jacobi iteration
updates the entire $u$ vector based on the old guesses; Gauss-Seidel
sweeps through the labels and updates them in a fixed order, using the
most recent guesses in each update; and Gauss-Southwell chooses the
next label to update adaptively, based on the size of a corresponding
residual element.  In machine learning, these are known as
{\em label propagation} methods, though label propagation methods
generally include an additional rounding operation to turn soft labels into
hard labels at each step.
The convergence of such iterations depends strongly on the
nature of the similarity graph: if it is ``tightly connected,'' the
iterations converge quickly, while the iterations may converge much
more slowly if the connectivity is relatively sparse.  We will return
to this point later.

\section{From Laplacians to kernels}

We have seen an expression of the form
\[
  u = -L_{uu}^{-1} L_{uy} y
\]
once before, in our initial discussion of Gaussian processes.  There,
instead of the graph Laplacian, we saw the {\em precision matrix}
(the inverse of the covariance).  We would therefore {\em like} to say
that $L^{-1}$ is a kernel.  Of course, we have to worry about
a slight caveat: $L$ is not invertible!  Hence, while we can still
define a kernel associated with $L$, we will have to use a
conditionally positive definite kernel associated with the
{\em pseudo-inverse} $L^\dagger$.

\subsection{The pseudoinverse as a kernel}

The Laplacian pseudo-inverse is $L^\dagger$, corresponding to the
minimal-norm least-squares solution to linear systems with $L$.  In
terms of the eigendecomposition $L = Q \Lambda Q^T$, the
pseudo-inverse is $L^\dagger = Q \Lambda^\dagger Q^T$ where
$\lambda_i^\dagger = \lambda_i^{-1}$ for nonzero $\lambda_i$, and is
zero otherwise.  Note that $LL^\dagger = L^\dagger L = J$ where $J$ is
the centering matrix $J = I-ee^T/n$.

Indeed, we can think of the soft label
problem as a kernel method involving the (conditionally positive
definite) kernel matrix $L^\dagger$; that is,
\[
  u = [L^\dagger]_{uy} c + \mu e
\]
where the weight vector $c$ is given by
\[
  \begin{bmatrix}
    [L^\dagger]_{yy} & e \\
    e^T & 0
  \end{bmatrix}
  \begin{bmatrix} c \\ \mu \end{bmatrix} =
  \begin{bmatrix} y \\ 0 \end{bmatrix}.
\]
To see this is equivalent to what we wrote before, we observe that
\begin{align*}
  L_{uu} [L^\dagger]_{uy} + L_{uy} [L^\dagger]_{yy} &= J_{uy} = ee^T/n \\
  L_{uu} e + L_{uy} e &= 0
\end{align*}
Because $e^T c = 0$ by construction, we therefore have
\[
  L_{uu} u + L_{uy} y = 
  L_{uu} ([L^{\dagger}]_{uy} c + \mu e) +
  L_{uy} ([L^\dagger]_{yy} c + \mu e) = 0,
\]
which is indeed the equation that we used to define $u$ previously.

\subsection{Laplacian features}

It is also helpful to think about this kernel in terms of feature
vectors.  Let $\Psi^T = Q' \Lambda'^{-1/2}$, where $Q'$ and $\Lambda'$
are the parts of the eigendecomposition corresponding to the nonzero
eigenvalue, so that $L^\dagger = \Psi^T \Psi$.  The columns of $\Psi$
are the feature vectors in the graph associated with the kernel, and
the soft label function is equivalent to $x_i = \psi_i^T d + \mu$
where $d$ is the minimal norm vector such that $\Psi_y^T d + \mu e = y$.
To see that this is equivalent, consider the constrained optimization
\[
  \mbox{minimize } \frac{1}{2} \|d\|^2
  \mbox{ s.t.~} \Psi_y^T d + \mu e = y,
\]
and note that the KKT equations are
\[
  \begin{bmatrix}
    I & \Psi_y & 0 \\
    \Psi_y^T & 0 & e \\
    0 & e^T & 0
  \end{bmatrix}
  \begin{bmatrix} d \\ \lambda \\ \mu \end{bmatrix} =
  \begin{bmatrix} 0 \\ y \\ 0 \end{bmatrix}.
\]
Eliminating the first equation $d = -\Psi_y \lambda$ gives us
\[
  \begin{bmatrix}
    -\Psi_y^T \Psi_y & e \\
    e^T & 0
  \end{bmatrix}
  \begin{bmatrix} \lambda \\ \mu \end{bmatrix} =
  \begin{bmatrix} y \\ 0 \end{bmatrix},
\]
which we can rewrite as
\[
  \begin{bmatrix}
    [L^\dagger]_{yy} & e \\
    e^T & 0
  \end{bmatrix}
  \begin{bmatrix} -\lambda \\ \mu \end{bmatrix} =
  \begin{bmatrix} y \\ 0 \end{bmatrix}.
\]
This is the same system that we saw a moment ago, but with
$c = -\lambda$ reinterpreted as a vector of Lagrange multipliers.
Therefore, the minimal norm coefficient vector in the feature space
is $d = \Psi_y c$, which gives us the prediction
\[
  u = \Psi_u^T d + \mu e = \Psi_u^T \Psi_y c + \mu e =
  [L^\dagger]_{uy} c + \mu e.
\]

We will see the eigenvector features associated with the $L^\dagger$
kernel again next time when we address {\em unsupervised} learning
with graphs.

\section{Electrical analogies}

% Currents entering a node equal the currents leaving the node.
% Currents flowing out of i:
%   sum_j G_{ij} (V_i-V_j)
% Term is positive if V_i > V_j

So far, we have focused on a purely mathematical intuition for the
soft labeling problem.  But we can also consider a more physical
picture.  We will consider the flow of current through a resistor
network, which is a common choice in this business\footnote{%
  Other analogies involve pressure-driven flow through a pipe network
  or motion of a spring network.}
We suppose there are $n$ nodes connected by resistors.  At each node,
we have a voltage $v_i$, and on each resistor edge we have a
resistance $r_{ij}$.  There are two basic ingredients to the equations:
\begin{itemize}
\item A {\em constitutive law}: For a linear resistor,
  the current from $i$ to $j$ is
  \[
    I_{ij} = r_{ij}^{-1} (v_i-v_j).
  \]
\item A {\em balance law}: The total current leaving a node is zero, or
  \[
    \sum_{j} I_{ij} = 0.
  \]
\end{itemize}
Putting these two ingredients together gives us the system
\[
  \sum_{j \in N_i} r_{ij}^{-1} (v_i-v_j) = 0
\]
at each node $i$ for which we do not explicitly control the voltage
(by attaching the node to ground or a voltage supply) or inject a
current.  This gives us a weighted Laplacian linear system,
where the Laplacian is known as the {\em conductance matrix} in
circuit theory, and the edge weights $a_{ij}$ are the
element conductances (inverse resistances\footnote{%
  In a circuit theory class, I would write the conductances as $g_{ij}
  = r_{ij}^{-1}$.  But to maintain notational consistency with the
  rest of the lecture, we will use $a_{ij}$ here.
}).
Hence, the soft labeling problem is equivalent to drawing a resistive
circuit network and attaching some nodes to a unit voltage supply (the
examples labeled 1) and others attached to ground (the examples
labeled 0).  The intuition is that nodes that are connected by
low-resistance edges or paths tend to have similar voltages.  The
Laplacian quadratic form is associated with resistive power loss.

Whether the analogy to circuit theory provides insight or not probably
depends on your background.  But the analogy is sufficiently widely
used that it is worth knowing about, whether or not you find it
provides you with any personal intuition.

\section{Kernels and distances}

Positive definite kernels define inner products in a feature space,
and inner products define a Euclidean distance structure.  That is, if
$\psi$ is a feature map for a kernel on a space $\mathcal{X}$, then
\begin{align*}
  \|\psi(x)-\psi(y)\|^2
  &= \psi(x)^T \psi(x) - 2 \psi(x)^T \psi(y) + \psi(y)^T \psi(y) \\
  & = k(x,x) - 2 k(x,y) + k(y,y).
\end{align*}
In the positive definite case, we can therefore use the kernel to
define a squared distance on $\mathcal{X}$:
\[
  d(x,y)^2 = k(x,x) - 2 k(x,y) + k(y,y),
\]
and this distance satisfies all the properties that a distance is supposed
to satisfy (positivity, symmetry, and the triangle inequality).

Of course, the kernel associated with the graph Laplacian is only
positive {\em semi}-definite because of the null vector.  The usual
hazard for semi-definite kernel functions is that we might have
distinct points in $\mathcal{X}$ with the same feature vector, and a
distance between two points is supposed to be nonzero if the points
are distinct.  We do not have to worry about this problem with the
Laplacian kernel, though, as the construction in this case looks like
\[
  d_{ij}^2 = (e_i-e_j)^T L^\dagger (e_i-e_j);
\]
and since the vectors $e_i-e_j$ are orthogonal to the null vector of
all ones, this quantity will be positive for all $i \neq j$.

We sometimes call $d_{ij}^2$ the {\em resistance distance}, since in the
electrical analogy it corresponds to the effective resistance between
nodes $i$ and $j$ summarized over all possible network paths.
In the physical analogy, the current balance law holds in the
following generalized sense: if $S$ is the set of nodes for which we
have specified voltages (label information), then for any $i \not \in S$,
\[
  \sum_{j \in S} d_{ij}^{-2} (v_i-v_j) = 0;
\]
we can rewrite this as
\[
  v_i = \frac{\sum_{j \in S} d_{ij}^{-2} v_j}{\sum_{j \in S} d_{ij}^{-2}};
\]
that is, the computed value at node $i$ is a weighted average of the
known values, where the weights are proportional to the inverse-square
distances.  This formula for the soft labeling function works even
with other kernel functions --- though, of course, we lose the circuit
analogy!

\section{The heat kernel}

So far, we have focused on the inverse Laplacian graph kernel.
However, this is not the only choice!  Another kernel that we can use
for many of the same purposes is the {\em heat kernel}, which is given
by $\exp(-t L)$.  The parameter is associated with time, and the
entries of $\exp(-t L)$ can be interpreted in terms of the diffusion
of heat from a source at $i$ to a target at $j$ within time $t$.
Alternately, the entries $\exp(-t L)_{ij}$ can be interpreted as the
probability that a continuous random walk starting from $i$ will be
at $j$ at time $t$.

\section{Extending to multi-class learning}

So far, we have focused on the two-class case with 0-1 labels.
For the more general case where we want $k$ different classes,
we use the same technique applied to $k$ indicator vectors,
one for each class.  That is, we replace the vector $x \in \bbR^n$ with
the matrix $X \in \bbR^{n \times k}$.  In the hard label case, we let
$x_{ik}$ be one if $i$ belongs to class $k$ and zero otherwise.
In the soft label case, we assign node $i$ to the class $k$ for which
$x_{ik}$ is maximal.  We also have that $\sum_{k} x_{ik} = 1$, and
so sometimes $x_{ik}$ is interpreted as the probability that node $i$
belongs to class $k$.

\section{The Laplace solver building block}

We conclude this lecture with a brief discussion of the landscape of
methods for solving Laplacian linear systems.

For small systems --- up to a few thousand nodes --- there is not much
to discuss.  In these cases, forming and factoring the Laplacian
matrix as a dense matrix is usually fine, and requires little thought
or care.  Past a few thousand nodes, though, the $O(n^3)$ cost of
a dense matrix factorization becomes prohibitive.  In this case,
we can either
\begin{itemize}
\item Use a {\em sparse direct} method that computes a factorization
  in less than $O(n^3)$ time, or
\item Use an iterative solver.
\end{itemize}
Of course, the two methods are not mutually exclusive, and we often
use approximate factorizations as {\em preconditioners} for iterative
methods.  But it is important to recognize that many graphs are
{\em either} well suited to iterative methods {\em or} well suited
to sparse direct solvers.  The key distinction is whether the graph
can be separated by relatively small cuts (a problem we will consider
in the next lecture).

When a graph can be partitioned with a small cut, we can try to solve
it by a {\em divide and conquer} approach.  Suppose that there is a
small {\em vertex separator} that partitions the graph into two
roughly-equal size pieces.  If we label the two separate pieces first
and then put the separator at the end, then we can write the Laplacian
system in block form as
\[
  L = 
  \begin{bmatrix}
    L_{11} & 0 & L_{13} \\
    0 & L_{22} & L_{23} \\
    L_{31} & L_{32} & L_{33}
  \end{bmatrix}.
\]
The structure comes from the observation that the degrees of freedom
in the two pieces (block 1 and block 2) are not directly connected.
Block Gaussian elimination on the system gives us
\begin{align*}
  S &= L_{33} - L_{31} L_{11}^{-1} L_{13} - L_{32} L_{22}^{-1} L_{23} \\
  S x_3 &= b_3 - L_{31} L_{11}^{-1} b_1 - L_{32} L_{22}^{-1} b_2 \\
  L_{22} x_2 &= b_2 - L_{23} x_3 \\
  L_{11} x_1 &= b_1 - L_{13} x_3
\end{align*}
Hence, if we can quickly solve systems with $L_{11}$ and $L_{22}$,
then we can form and solve a much smaller {\em Schur complement}
system to couple them together.  The {\em nested dissection} approach
applies this idea recursively, and gives us a very fast solver
{\em if we can find small separators}.

Of course, the extreme case of small separators is when we have
a tree.  In this case we can produce very fast solvers that run in
linear time in the matrix size.  One way to see this is in the
electrical network analogy: we can compute the resistance between
any pair of nodes quickly because it is just the sum of the
resistances along the unique path between those nodes!  More
generally, graphs that are associated with nearest neighbor
connectivity in 2D (or sometimes 3D) tend to have small
{\em tree width}, and are good for sparse solvers.  There are good
sparse solvers in the world, and I do not recommend writing your own.
But it is important to know what graphs are well suited to sparse solvers.

The opposite extreme is when there are no small separators.
In this case, though, the smallest nonzero eigenvalue of the Laplacian
is usually far from zero, so that the condition number of the
Laplacian system is not too large.  This is exactly the situation
in which standard iterative methods work well.

\end{document}
